\documentclass[12pt, a4paper]{article}

\usepackage{xeCJK} %auto load "fontspec" package
	\setmainfont{Times New Roman} %設主字型為 "Times New Roman"
	\defaultCJKfontfeatures{AutoFakeBold = 6, AutoFakeSlant = 0.4} 
		%以後不用再設定粗斜
		%AutoFakeBold設定粗體字要多粗
		%AutoFakeSlant設定斜體字要多斜,範圍-0.999到0.999,負值為往左斜
	\setCJKmainfont{新細明體}	%設中文字體字型為 "新細明體"
		%以下四行非必要,但對於切換字型蠻好用的
	\newCJKfontfamily \Kai{標楷體}       	%定義指令\Kai則切換成標楷體
	\newCJKfontfamily \Hei{微軟正黑體}   	%定義指令\Hei則切換成正黑體
	\newCJKfontfamily \NewMing{新細明體} 	%定義指令\NewMing則切換成新細明體
	\XeTeXlinebreaklocale "zh"             	%這兩行一定要加,中文才能自動換行
	\XeTeXlinebreakskip = 0pt plus 1pt     	%這兩行一定要加,中文才能自動換行

\usepackage[margin = 1in]{geometry} %1in=25.4mm
\usepackage[UKenglish]{babel}
\usepackage{setspace}
\usepackage{fancyhdr}

%\usepackage{textcomp} %do not load for xelatex because it read T1 font while xelatex read UTF-8
\usepackage{graphicx}
\usepackage{xcolor}
\usepackage{adjmulticol}
\usepackage{wrapfig}
\usepackage{subcaption}
\usepackage{array}

\usepackage{gensymb} 
\usepackage{amsmath}
\usepackage{amssymb}
\usepackage{amsfonts}

\begin{document}
\begin{center} \Large \textbf{如何使用 Xe\LaTeX 輸出中文PDF} \end{center}

\subsection*{設定}
1. 打開 ``Configure Texmaker''\\
2. 修改 ``Quick Build'' 為 ``XeLatex + View PDF''\\
3. 修改 ``Editor Font Encoding'' 為 UTF-8\\
4. 安裝 ``xecjk'' package 及 ``l3kernel'' package

\subsection*{載入套件}
輸入 \textbackslash usepackage\{xeCJK\} ,此套件會自動載入 ``fontspec''\\
輸入 \textbackslash setmainfont\{`mainfont'\} ,設定主字型為 `mainfont',`mainfont' 可為 `Times New Roman' 等系統字體\\
輸入 \textbackslash defaultCJKfontfeatures\{AutoFakeBold=6, AutoFakeSlant=0.4\} ,設定默認粗斜,\linebreak
AutoFakeBold 設定粗體字要多粗、AutoFakeSlant 設定斜體字要多斜,範圍由 -0.999到 0.999,負值為往左斜,如所選字型已有定義粗斜則可不用加,新細明體因沒有定義所以必需加
輸入 \textbackslash setCJKmainfont\{`字體'\} ,設中文字型為 `字體',`字體'可為`新細明體'等系統字體\\
輸入 \textbackslash XeTeXlinebreaklocale ``zh" 及 \textbackslash XeTeXlinebreakskip = 0pt plus 1pt,這兩行一定要加,中文才能自動換行\\
以下四行非必要,但對於切換字型蠻好用的\\
\textbackslash newCJKfontfamily \textbackslash Kai\{標楷體\} ,定義指令 \textbackslash Kai 則切換成標楷體\\
\textbackslash newCJKfontfamily \textbackslash Hei\{微軟正黑體\} ,定義指令 \textbackslash Hei 則切換成正黑體\\
\textbackslash newCJKfontfamily \textbackslash NewMing\{新細明體\} ,定義指令 \textbackslash NewMing 則切換成新細明體

\subsection*{範例}
\textbackslash documentclass[12pt, a4paper]\{article\}		\\[1ex]
\textbackslash usepackage\{xeCJK\}	\\
\textbackslash setmainfont\{Times New Roman\}		\\[1ex]
\textbackslash defaultCJKfontfeatures\{AutoFakeBold=6, AutoFakeSlant=0.4\}	\\
\textbackslash setCJKmainfont\{新細明體\}	\\
\textbackslash newCJKfontfamily \textbackslash Kai\{標楷體\}	\\       	
\textbackslash newCJKfontfamily \textbackslash Hei\{微軟正黑體\}	\\  
\textbackslash newCJKfontfamily \textbackslash NewMing\{新細明體\} 	\\
\textbackslash XeTeXlinebreaklocale "zh"            \\
\textbackslash XeTeXlinebreakskip = 0pt plus 1pt  \\[1ex]
\textbackslash begin\{document\}	\\
\hspace*{1em} 範例\\
\textbackslash end\{document\}
\end{document}