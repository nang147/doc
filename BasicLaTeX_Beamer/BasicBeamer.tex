\documentclass{beamer} 

\usepackage{adjmulticol}

%\usetheme{Warsaw} %there are many themes that can use
%\usecolortheme{crane} %colortheme instead of theme, which can co-exist with theme
%\setbeamercolor{background canvas}{bg=pink} %set background color	

\begin{document}
\begin{frame}
\title{Basic Beamer}
%\subtitle{Subtitle Here}
\author{}
\date{}
\institute{institute}
\titlepage %build a titlepage with the information above
\end{frame}

\begin{frame}{Required packages}
\begin{enumerate} %[A] you can add this to change the style of numbering
\item beamer
\item pgf
\item etoolbox
\item ms
\item url
\item sansmathaccent
\item filehook
\item translator
\end{enumerate}
\end{frame}

\begin{frame}[t]{Themes} \vspace{3mm} %\vspace are used to add verticle space after title
You can find beamer themes on \textbf{beamer-theme-matrix}\\
You can apply color theme together with theme\\
You may be able to create your own color theme by the command \textbackslash  definecolor\\
You can set background color by using command\\
\textbackslash setbeamercolor\{background canvas\}\{bg=color\}
\end{frame}

\begin{frame}[t]{Basic}
\setbeamercovered{transparent = 15} %use to control the transparacy of hidden text
\only<1>{ \begin{block}{Create a block}
Line one of this block\\
Line two of this block
\end{block} 
\textcolor{blue}{You can change text color}\\}
One frame can contain multiple slides.\\
You can also set details that appear on which slide in a frame by using the command \textbackslash only.\linebreak
\only<1>{You can only break this line by \textbackslash linebreak, I don't know why}\\
\only<2>{This will only show in slide two, it will completely delete\\} %notice the use of \\
\onslide<2>{You can also use onslide to do so, but it is only hidden\\}
\onslide<3>{Note that onslide is only to hide the text, which means to make them transparent}
\only<3>{This will only show in slide three}
\only<1-2>{This will only show in slides one and two, you can see this is affected by hidden text\\}
\only<1->{This will show in the slides after slide one\\}
\end{frame}

\begin{frame}{Columns: method 1}
\begin{columns}[onlytextwidth] %[onlytextwidth] help alignment
\column{0.5 \textwidth}
%\column{definw width}, here is that only text width is this column and there are two columns each occupy 0.5 textwidth
xxx
\column{0.5 \textwidth}
xxx
%you can \includegraphics[scale=•]{•} to add picture
\end{columns}
\end{frame}

\begin{frame}{Columns: method 2 using external package}
\begin{multicols}{3}
column 1 \\ column 2 \\ column 3 
\end{multicols}
\begin{enumerate}
\begin{multicols}{2}
\item This is from package "adjmulticol"
\item multicols can be used under enumerate environment
\end{multicols}
\end{enumerate}
\end{frame}
\end{document}